% Options for packages loaded elsewhere
\PassOptionsToPackage{unicode}{hyperref}
\PassOptionsToPackage{hyphens}{url}
%
\documentclass[
  english,
  man]{apa6}
\usepackage{amsmath,amssymb}
\usepackage{lmodern}
\usepackage{ifxetex,ifluatex}
\ifnum 0\ifxetex 1\fi\ifluatex 1\fi=0 % if pdftex
  \usepackage[T1]{fontenc}
  \usepackage[utf8]{inputenc}
  \usepackage{textcomp} % provide euro and other symbols
\else % if luatex or xetex
  \usepackage{unicode-math}
  \defaultfontfeatures{Scale=MatchLowercase}
  \defaultfontfeatures[\rmfamily]{Ligatures=TeX,Scale=1}
\fi
% Use upquote if available, for straight quotes in verbatim environments
\IfFileExists{upquote.sty}{\usepackage{upquote}}{}
\IfFileExists{microtype.sty}{% use microtype if available
  \usepackage[]{microtype}
  \UseMicrotypeSet[protrusion]{basicmath} % disable protrusion for tt fonts
}{}
\makeatletter
\@ifundefined{KOMAClassName}{% if non-KOMA class
  \IfFileExists{parskip.sty}{%
    \usepackage{parskip}
  }{% else
    \setlength{\parindent}{0pt}
    \setlength{\parskip}{6pt plus 2pt minus 1pt}}
}{% if KOMA class
  \KOMAoptions{parskip=half}}
\makeatother
\usepackage{xcolor}
\IfFileExists{xurl.sty}{\usepackage{xurl}}{} % add URL line breaks if available
\IfFileExists{bookmark.sty}{\usepackage{bookmark}}{\usepackage{hyperref}}
\hypersetup{
  pdftitle={Item Characteristic Curves Generated from CTT Item Statistics},
  pdfauthor={Diego Figueiras1 \& John T. Kulas1},
  pdflang={en-EN},
  pdfkeywords={keywords},
  hidelinks,
  pdfcreator={LaTeX via pandoc}}
\urlstyle{same} % disable monospaced font for URLs
\usepackage{graphicx}
\makeatletter
\def\maxwidth{\ifdim\Gin@nat@width>\linewidth\linewidth\else\Gin@nat@width\fi}
\def\maxheight{\ifdim\Gin@nat@height>\textheight\textheight\else\Gin@nat@height\fi}
\makeatother
% Scale images if necessary, so that they will not overflow the page
% margins by default, and it is still possible to overwrite the defaults
% using explicit options in \includegraphics[width, height, ...]{}
\setkeys{Gin}{width=\maxwidth,height=\maxheight,keepaspectratio}
% Set default figure placement to htbp
\makeatletter
\def\fps@figure{htbp}
\makeatother
\setlength{\emergencystretch}{3em} % prevent overfull lines
\providecommand{\tightlist}{%
  \setlength{\itemsep}{0pt}\setlength{\parskip}{0pt}}
\setcounter{secnumdepth}{-\maxdimen} % remove section numbering
% Make \paragraph and \subparagraph free-standing
\ifx\paragraph\undefined\else
  \let\oldparagraph\paragraph
  \renewcommand{\paragraph}[1]{\oldparagraph{#1}\mbox{}}
\fi
\ifx\subparagraph\undefined\else
  \let\oldsubparagraph\subparagraph
  \renewcommand{\subparagraph}[1]{\oldsubparagraph{#1}\mbox{}}
\fi
% Manuscript styling
\usepackage{upgreek}
\captionsetup{font=singlespacing,justification=justified}

% Table formatting
\usepackage{longtable}
\usepackage{lscape}
% \usepackage[counterclockwise]{rotating}   % Landscape page setup for large tables
\usepackage{multirow}		% Table styling
\usepackage{tabularx}		% Control Column width
\usepackage[flushleft]{threeparttable}	% Allows for three part tables with a specified notes section
\usepackage{threeparttablex}            % Lets threeparttable work with longtable

% Create new environments so endfloat can handle them
% \newenvironment{ltable}
%   {\begin{landscape}\centering\begin{threeparttable}}
%   {\end{threeparttable}\end{landscape}}
\newenvironment{lltable}{\begin{landscape}\centering\begin{ThreePartTable}}{\end{ThreePartTable}\end{landscape}}

% Enables adjusting longtable caption width to table width
% Solution found at http://golatex.de/longtable-mit-caption-so-breit-wie-die-tabelle-t15767.html
\makeatletter
\newcommand\LastLTentrywidth{1em}
\newlength\longtablewidth
\setlength{\longtablewidth}{1in}
\newcommand{\getlongtablewidth}{\begingroup \ifcsname LT@\roman{LT@tables}\endcsname \global\longtablewidth=0pt \renewcommand{\LT@entry}[2]{\global\advance\longtablewidth by ##2\relax\gdef\LastLTentrywidth{##2}}\@nameuse{LT@\roman{LT@tables}} \fi \endgroup}

% \setlength{\parindent}{0.5in}
% \setlength{\parskip}{0pt plus 0pt minus 0pt}

% \usepackage{etoolbox}
\makeatletter
\patchcmd{\HyOrg@maketitle}
  {\section{\normalfont\normalsize\abstractname}}
  {\section*{\normalfont\normalsize\abstractname}}
  {}{\typeout{Failed to patch abstract.}}
\patchcmd{\HyOrg@maketitle}
  {\section{\protect\normalfont{\@title}}}
  {\section*{\protect\normalfont{\@title}}}
  {}{\typeout{Failed to patch title.}}
\makeatother
\shorttitle{CTT ICCs}
\keywords{keywords\newline\indent Word count: X}
\DeclareDelayedFloatFlavor{ThreePartTable}{table}
\DeclareDelayedFloatFlavor{lltable}{table}
\DeclareDelayedFloatFlavor*{longtable}{table}
\makeatletter
\renewcommand{\efloat@iwrite}[1]{\immediate\expandafter\protected@write\csname efloat@post#1\endcsname{}}
\makeatother
\usepackage{csquotes}
\ifxetex
  % Load polyglossia as late as possible: uses bidi with RTL langages (e.g. Hebrew, Arabic)
  \usepackage{polyglossia}
  \setmainlanguage[]{english}
\else
  \usepackage[main=english]{babel}
% get rid of language-specific shorthands (see #6817):
\let\LanguageShortHands\languageshorthands
\def\languageshorthands#1{}
\fi
\ifluatex
  \usepackage{selnolig}  % disable illegal ligatures
\fi
\newlength{\cslhangindent}
\setlength{\cslhangindent}{1.5em}
\newlength{\csllabelwidth}
\setlength{\csllabelwidth}{3em}
\newenvironment{CSLReferences}[2] % #1 hanging-ident, #2 entry spacing
 {% don't indent paragraphs
  \setlength{\parindent}{0pt}
  % turn on hanging indent if param 1 is 1
  \ifodd #1 \everypar{\setlength{\hangindent}{\cslhangindent}}\ignorespaces\fi
  % set entry spacing
  \ifnum #2 > 0
  \setlength{\parskip}{#2\baselineskip}
  \fi
 }%
 {}
\usepackage{calc}
\newcommand{\CSLBlock}[1]{#1\hfill\break}
\newcommand{\CSLLeftMargin}[1]{\parbox[t]{\csllabelwidth}{#1}}
\newcommand{\CSLRightInline}[1]{\parbox[t]{\linewidth - \csllabelwidth}{#1}\break}
\newcommand{\CSLIndent}[1]{\hspace{\cslhangindent}#1}

\title{Item Characteristic Curves Generated from CTT Item Statistics}
\author{Diego Figueiras\textsuperscript{1} \& John T. Kulas\textsuperscript{1}}
\date{}


\authornote{

Add complete departmental affiliations for each author here. Each new line herein must be indented, like this line.

Enter author note here.

The authors made the following contributions. John T. Kulas: .

Correspondence concerning this article should be addressed to Diego Figueiras, Dickson Hall 226. E-mail: \href{mailto:figueirasd1@montclair.edu}{\nolinkurl{figueirasd1@montclair.edu}}

}

\affiliation{\vspace{0.5cm}\textsuperscript{1} Montclair State University}

\begin{document}
\maketitle

\hypertarget{introduction}{%
\section{Introduction}\label{introduction}}

Item characteristic curves are referenced by psychometricians as visual indicators of important attributes of assessment items - most commonly \emph{difficulty} and/or \emph{discrimination}. Within these visual presentations the x-axis ranges along ``trait'' levels (by convention annotated with the greek \(\theta\)), whereas the y-axis displays probabilities of responding to the item within a given response category. In the context of true tests, the response categories are binary, and the y-axis probability refers to the likelihood of a ``correct'' response. From this orientation, each item within a test can be represented with a visualized ``curve.'' By looking at it, we can know the likelihood with which respondents of any trait level would answer any item correctly. If the curve is leaning towards the lower end of the trait level, this indicates that it is easy to answer the item correctly. On the contrary, if the curve is leaning towards the higher end of the trait level, this indicates that the item is difficult. If the curve is steep, this indicates high discrimination among respondents; if it is flat, it indicates no discrimination.

\begin{figure}
\centering
\includegraphics{ICC_project_files/figure-latex/unnamed-chunk-1-1.pdf}
\caption{\label{fig:unnamed-chunk-1}Item characteristic curves primarily reflecting differences in discrimination.}
\end{figure}

Psychometricians who examine ICCs usually do it using Item Response Theory and Rasch models to get the parameters necessary to plot the curves. In a 2PL model, these would be item difficulty and item discrimination. Item difficulty is the necessary trait level for a respondent to have a 50/50 chance to answer the item correctly. Item discrimination is the degree to which an item can differentiate among individuals with low and high levels of the trait. From a Classical Test Theory (CTT) frame of thinking, the difficulty of an item is determined by looking at the p-values of the items, while discrimination is determined by checking the Cronbach alpha and the corrected item total correlations. Psychometricians who look at these CTT parameters don't typically use them to plot ICCs.There is no reason for them not to, since ICCs based on CTT parameters could provide information as valuable as those based on IRT or Rasch without the need of being familiar with these models and with how to compute the necessary estimates. Fan states in summary that IRT and CTT ``\ldots{} framework produce very similar item and person statistics'' (p.379).

There is research that shows that there is little difference between the parameters of both frameworks.@hambleton1993comparison concluded that ``no study provides enough empirical evidence on the extent of disparity between the two frameworks and the superiority of IRT over CTT despite the theoretical differences.''

Fan (1998) conducted a study to empirically test the differences between the two frameworks. According to him, ``The findings here simply show that the two measurement frameworks produced very similar item and person statistics both in terms of the comparability of item and person statistics between the two frameworks and in terms of the degree of invariance of item statistics from the two competing measurement frameworks.'' In his study, Fan (1998) looked at the correlations between ability estimates and item difficulty in CTT and all three IRT models. These correlations were very high, between high .80 and low .90. As of item discrimination, correlations were moderate to high, with only a few being very low.

He also looked at the item invariance for all models. In theory, the major advantage of IRT models over CTT is that the latter has a circular dependency between the item and person statistics, while IRT has no such dependency, which means that the item parameters don't depend on the sample and the person parameters don't depend on the set of items. This property of invariance is very important, since item estimates can be used regardless of the sample you are giving the test or assessment to. An item will always have the same level of difficulty regardless of who is responding, for example.

What Fan (1998) got on his study, however, shows empirical evidence against this supposed advantage of IRT against CTT. The CTT item difficulty and discrimination degrees of invariance were highly correlated with those of IRT, indicating that they were highly comparable.

Lord (2012) described a function that approximates the relationship between IRT parameters and the CTT discrimination index of an item-test biserial correlation:

\[a_i\cong \frac{r_i}{\sqrt{1-r_i^2}}\]

This formula wasn't intended for practical purposes but rather to assist in the conceptual comprehension of the discrimination parameter in IRT for people who were more familiar with CTT procedures. In an effort to move from the conceptual to a practical application, Kulas, Smith, and Xu (2017) proposed a modification that minimized the average residual (either \(a_i\) or \(r_i\), where \(r_i\) is the \emph{corrected} item-total \emph{point-biserial} correlation).

Simulations identified systematic slope and inflection differences across item with differing item difficulty values, so the formula was further changed to include the following modifiers This revised formula is used in the current presentation:

\[\hat{a_i}\cong[(.51 + .02z_g + .3z_g^2)r]+[(.57 - .009z_g + .19z_g^2)\frac{e^r-e^{-r}}{e-e^r}]\]

Where \(g\) is the absolute deviation from 50\% responding an item correctly and 50\% responding incorrectly (e.g., a ``p-value'' of .5). \(Z_g\) is the standard normal deviation associated with \(g\). The transformation of the standard p-value was recommended in order to scale this index along an interval-level metric more directly anaologous to the IRT \emph{b} parameter. Figure XX visualizes the re-specifications of Lord's formula at p-values (difficulty) of .5, .3 (or .7), and .1 (or .9).

\begin{figure}
\centering
\includegraphics{ICC_project_files/figure-latex/acorrected-1.pdf}
\caption{\label{fig:acorrected}Empricially derived functional relationship between the IRT \emph{a} parameter and the CTT corrected-item total correlation as a function of item difficulty (p-value; solid = .5, dashed = .3/.7, dotted = .1/.9).}
\end{figure}

As we can see, the higher the corrected item-total correlations, the higher the estimated IRT a-parameter (discrimination). Also, as the p-values (difficulty) deviates from 0, the relationship between the estimated IRT a-parameter and the corrected item-total correlations becomes stronger.

Practitioners and researchers that don't use IRT or Rasch models and instead opt to follow a CTT philosophy would benefit from having ICCs that use CTT statistics. This study intends to show evidence of the overlapping nature of CTT and IRT parameters when it comes to plotting ICCs.

\hypertarget{study-1---visual-of-discrimination-relationship}{%
\section{Study 1 - Visual of discrimination relationship}\label{study-1---visual-of-discrimination-relationship}}

The purpose of study 1 is to look at the visualizations resulting from Kulas, Smith, and Xu (2017) formula on simulated data. We hypothesize that the relationship between the estimated IRT a-parameter and the corrected item-total correlations will be stronger as the later deviates from 0, which would mean that the item has more discrimination.

\hypertarget{procedure-and-methods}{%
\subsection{Procedure and methods}\label{procedure-and-methods}}

We simulated data using Han (2007) software. Our sample was 10,000 observations, with a mean of 0 and a standard deviation of 1. The number of items were 100, with response categories of either correct or incorrect (1 and 0).The mean for the a parameter was 2, and the standard deviation 0.8. The mean for parameter b was 0 and the standard deviation 0.5.

\hypertarget{results}{%
\subsection{Results}\label{results}}

\begin{figure}
\centering
\includegraphics{ICC_project_files/figure-latex/acorrected simulation-1.pdf}
\caption{(\#fig:acorrected simulation)Functional relationship between the IRT \emph{a} parameter and the CTT corrected-item total correlation as a function of item difficulty (p-value; solid = .3, dashed = .9, dotted = .12).}
\end{figure}

\hypertarget{study-2---item-characteristic-curves-comparisons.}{%
\section{Study 2 - Item Characteristic Curves comparisons.}\label{study-2---item-characteristic-curves-comparisons.}}

The purpose of study 2 is to simulates a lot of test data and then generate ICCs based on the IRT model and then we compare that to our CTT estimates.

\hypertarget{procedure-and-materials}{%
\subsection{Procedure and materials}\label{procedure-and-materials}}

The same simulated data as in study 1 was used. The mirt package was used to compute the IRT statistics. The blue curves were plotted using 2PL IRT parameters (a and b), while the red curves were plotted using CTT parameters (p-values and corrected item-total correlations, modifying them with Kulas, Smith, and Xu (2017) formulas).

\hypertarget{results-1}{%
\subsection{Results}\label{results-1}}

We used R {[}Version 4.1.1; R Core Team (2020){]} and the R-packages \emph{\}ape} {[}@\}R-ape{]}, \emph{dplyr} {[}Version 1.0.7; Wickham, François, Henry, and Müller (2021){]}, \emph{DT} {[}Version 0.19; Xie, Cheng, and Tan (2021){]}, \emph{forcats} {[}Version 0.5.1; Wickham (2021a){]}, \emph{formattable} (Ren \& Russell, 2021), \emph{geiger} {[}Version 2.0.7; Alfaro et al. (2009); Eastman, Alfaro, Joyce, Hipp, and Harmon (2011); Slater et al. (2012); Harmon, Weir, Brock, Glor, and Challenger (2008); Pennell et al. (2014){]}, \emph{ggplot2} {[}Version 3.3.5; Wickham (2016){]}, \emph{gridExtra} {[}Version 2.3; Auguie (2017){]}, \emph{irtplay} {[}Version 1.6.2; Lim (2020){]}, \emph{jpeg} {[}Version 0.1.9; Urbanek (2021){]}, \emph{knitr} {[}Version 1.34; Xie (2015){]}, \emph{lattice} {[}Version 0.20.44; Sarkar (2008); Sarkar and Andrews (2019){]}, \emph{latticeExtra} {[}Version 0.6.29; Sarkar and Andrews (2019){]}, \emph{markdown} {[}Version 1.1; Allaire, Horner, Xie, Marti, and Porte (2019); Xie, Allaire, and Grolemund (2018); Xie, Dervieux, and Riederer (2020){]}, \emph{mirt} {[}Version 1.34; Chalmers (2012){]}, \emph{officer} (Gohel, 2021), \emph{papaja} {[}Version 0.1.0.9997; Aust and Barth (2020){]}, \emph{pdftools} {[}Version 3.0.1; Ooms (2021){]}, \emph{psych} {[}Version 2.1.9; Revelle (2021){]}, \emph{purrr} {[}Version 0.3.4; Henry and Wickham (2020){]}, \emph{readr} {[}Version 2.0.1; Wickham and Hester (2021){]}, \emph{readxl} {[}Version 1.3.1; Wickham and Bryan (2019){]}, \emph{reticulate} {[}Version 1.22; Ushey, Allaire, and Tang (2021){]}, \emph{rmarkdown} {[}Version 2.11; Xie, Allaire, and Grolemund (2018); Xie, Dervieux, and Riederer (2020){]}, \emph{shiny} {[}Version 1.7.0; Chang et al. (2021){]}, \emph{stringr} {[}Version 1.4.0; Wickham (2019){]}, \emph{tibble} {[}Version 3.1.4; Müller and Wickham (2021){]}, \emph{tidyr} {[}Version 1.1.3; Wickham (2021b){]}, \emph{tidyverse} {[}Version 1.3.1; Wickham et al. (2019){]}, and \emph{tinytex} {[}Version 0.33; Xie (2019){]} for all our analyses. The area between ICC's was calculated between CTT-derived and IRT-derived ICCs. The average difference for all 100 curves was 0.35.

\includegraphics{ICC_project_files/figure-latex/plotting-1.pdf}

\begin{figure}
\centering
\includegraphics{ICC_project_files/figure-latex/AUC-1.pdf}
\caption{\label{fig:AUC}ICCs using CTT parameters}
\end{figure}

\begin{figure}
\centering
\includegraphics{ICC_project_files/figure-latex/irt curves-1.pdf}
\caption{(\#fig:irt curves)ICCs using IRT parameters}
\end{figure}

\hypertarget{results-2}{%
\section{Results}\label{results-2}}

\hypertarget{discussion}{%
\section{Discussion}\label{discussion}}

\newpage

\hypertarget{references}{%
\section{References}\label{references}}

\begingroup
\setlength{\parindent}{-0.5in}
\setlength{\leftskip}{0.5in}

\hypertarget{refs}{}
\begin{CSLReferences}{1}{0}
\leavevmode\hypertarget{ref-R-geiger_a}{}%
Alfaro, M., Santini, F., Brock, C., Alamillo, H., Dornburg, A., Rabosky, D., \ldots{} Harmon, L. (2009). Nine exceptional radiations plus high turnover explain species diversity in jawed vertebrates. \emph{Proceedings of the National Academy of Sciences of the United States of America}, \emph{106}, 13410--13414.

\leavevmode\hypertarget{ref-R-markdown}{}%
Allaire, J., Horner, J., Xie, Y., Marti, V., \& Porte, N. (2019). \emph{Markdown: Render markdown with the c library 'sundown'}. Retrieved from \url{https://CRAN.R-project.org/package=markdown}

\leavevmode\hypertarget{ref-R-gridExtra}{}%
Auguie, B. (2017). \emph{gridExtra: Miscellaneous functions for "grid" graphics}. Retrieved from \url{https://CRAN.R-project.org/package=gridExtra}

\leavevmode\hypertarget{ref-R-papaja}{}%
Aust, F., \& Barth, M. (2020). \emph{{papaja}: {Create} {APA} manuscripts with {R Markdown}}. Retrieved from \url{https://github.com/crsh/papaja}

\leavevmode\hypertarget{ref-R-mirt}{}%
Chalmers, R. P. (2012). {mirt}: A multidimensional item response theory package for the {R} environment. \emph{Journal of Statistical Software}, \emph{48}(6), 1--29. \url{https://doi.org/10.18637/jss.v048.i06}

\leavevmode\hypertarget{ref-R-shiny}{}%
Chang, W., Cheng, J., Allaire, J., Sievert, C., Schloerke, B., Xie, Y., \ldots{} Borges, B. (2021). \emph{Shiny: Web application framework for r}. Retrieved from \url{https://CRAN.R-project.org/package=shiny}

\leavevmode\hypertarget{ref-R-geiger_b}{}%
Eastman, J., Alfaro, M., Joyce, P., Hipp, A., \& Harmon, L. (2011). A novel comparative method for identifying shifts in the rate of character evolution on trees. \emph{Evolution}, \emph{65}, 3578--3589.

\leavevmode\hypertarget{ref-fan1998item}{}%
Fan, X. (1998). Item response theory and classical test theory: An empirical comparison of their item/person statistics. \emph{Educational and Psychological Measurement}, \emph{58}(3), 357--381.

\leavevmode\hypertarget{ref-R-officer}{}%
Gohel, D. (2021). \emph{Officer: Manipulation of microsoft word and PowerPoint documents}. Retrieved from \url{https://CRAN.R-project.org/package=officer}

\leavevmode\hypertarget{ref-han2007wingen3}{}%
Han, K. (2007). WinGen3: Windows software that generates IRT parameters and item responses {[}computer program{]}. \emph{Amherst, MA: Center for Educational Assessment, University of Massachusetts Amherst}.

\leavevmode\hypertarget{ref-R-geiger_d}{}%
Harmon, L., Weir, J., Brock, C., Glor, R., \& Challenger, W. (2008). GEIGER: Investigating evolutionary radiations. \emph{Bioinformatics}, \emph{24}, 129--131.

\leavevmode\hypertarget{ref-R-purrr}{}%
Henry, L., \& Wickham, H. (2020). \emph{Purrr: Functional programming tools}. Retrieved from \url{https://CRAN.R-project.org/package=purrr}

\leavevmode\hypertarget{ref-kulas2017approximate}{}%
Kulas, J. T., Smith, J. A., \& Xu, H. (2017). Approximate functional relationship between IRT and CTT item discrimination indices: A simulation, validation, and practical extension of lord's (1980) formula. \emph{Journal of Applied Measurement}, \emph{18}(4), 393--407.

\leavevmode\hypertarget{ref-R-irtplay}{}%
Lim, H. (2020). \emph{Irtplay: Unidimensional item response theory modeling}. Retrieved from \url{https://CRAN.R-project.org/package=irtplay}

\leavevmode\hypertarget{ref-lord2012applications}{}%
Lord, F. M. (2012). \emph{Applications of item response theory to practical testing problems}. Routledge.

\leavevmode\hypertarget{ref-R-tibble}{}%
Müller, K., \& Wickham, H. (2021). \emph{Tibble: Simple data frames}. Retrieved from \url{https://CRAN.R-project.org/package=tibble}

\leavevmode\hypertarget{ref-R-pdftools}{}%
Ooms, J. (2021). \emph{Pdftools: Text extraction, rendering and converting of PDF documents}. Retrieved from \url{https://CRAN.R-project.org/package=pdftools}

\leavevmode\hypertarget{ref-R-geiger_e}{}%
Pennell, M., Eastman, J., Slater, G., Brown, J., Uyeda, J., Fitzjohn, R., \ldots{} Harmon, L. (2014). Geiger v2.0: An expanded suite of methods for fitting macroevolutionary models to phylogenetic trees. \emph{Bioinformatics}, \emph{30}, 2216--2218.

\leavevmode\hypertarget{ref-R-base}{}%
R Core Team. (2020). \emph{R: A language and environment for statistical computing}. Vienna, Austria: R Foundation for Statistical Computing. Retrieved from \url{https://www.R-project.org/}

\leavevmode\hypertarget{ref-R-formattable}{}%
Ren, K., \& Russell, K. (2021). \emph{Formattable: Create 'formattable' data structures}. Retrieved from \url{https://CRAN.R-project.org/package=formattable}

\leavevmode\hypertarget{ref-R-psych}{}%
Revelle, W. (2021). \emph{Psych: Procedures for psychological, psychometric, and personality research}. Evanston, Illinois: Northwestern University. Retrieved from \url{https://CRAN.R-project.org/package=psych}

\leavevmode\hypertarget{ref-R-lattice}{}%
Sarkar, D. (2008). \emph{Lattice: Multivariate data visualization with r}. New York: Springer. Retrieved from \url{http://lmdvr.r-forge.r-project.org}

\leavevmode\hypertarget{ref-R-latticeExtra}{}%
Sarkar, D., \& Andrews, F. (2019). \emph{latticeExtra: Extra graphical utilities based on lattice}. Retrieved from \url{https://CRAN.R-project.org/package=latticeExtra}

\leavevmode\hypertarget{ref-R-geiger_c}{}%
Slater, G., Harmon, L., Wegmann, D., Joyce, P., Revell, L., \& Alfaro, M. (2012). Fitting models of continuous trait evolution to incompletely sampled comparative data using approximate bayesian computation. \emph{Evolution}, \emph{66}, 752--762.

\leavevmode\hypertarget{ref-R-jpeg}{}%
Urbanek, S. (2021). \emph{Jpeg: Read and write JPEG images}. Retrieved from \url{https://CRAN.R-project.org/package=jpeg}

\leavevmode\hypertarget{ref-R-reticulate}{}%
Ushey, K., Allaire, J., \& Tang, Y. (2021). \emph{Reticulate: Interface to 'python'}. Retrieved from \url{https://CRAN.R-project.org/package=reticulate}

\leavevmode\hypertarget{ref-R-ggplot2}{}%
Wickham, H. (2016). \emph{ggplot2: Elegant graphics for data analysis}. Springer-Verlag New York. Retrieved from \url{https://ggplot2.tidyverse.org}

\leavevmode\hypertarget{ref-R-stringr}{}%
Wickham, H. (2019). \emph{Stringr: Simple, consistent wrappers for common string operations}. Retrieved from \url{https://CRAN.R-project.org/package=stringr}

\leavevmode\hypertarget{ref-R-forcats}{}%
Wickham, H. (2021a). \emph{Forcats: Tools for working with categorical variables (factors)}. Retrieved from \url{https://CRAN.R-project.org/package=forcats}

\leavevmode\hypertarget{ref-R-tidyr}{}%
Wickham, H. (2021b). \emph{Tidyr: Tidy messy data}. Retrieved from \url{https://CRAN.R-project.org/package=tidyr}

\leavevmode\hypertarget{ref-R-tidyverse}{}%
Wickham, H., Averick, M., Bryan, J., Chang, W., McGowan, L. D., François, R., \ldots{} Yutani, H. (2019). Welcome to the {tidyverse}. \emph{Journal of Open Source Software}, \emph{4}(43), 1686. \url{https://doi.org/10.21105/joss.01686}

\leavevmode\hypertarget{ref-R-readxl}{}%
Wickham, H., \& Bryan, J. (2019). \emph{Readxl: Read excel files}. Retrieved from \url{https://CRAN.R-project.org/package=readxl}

\leavevmode\hypertarget{ref-R-dplyr}{}%
Wickham, H., François, R., Henry, L., \& Müller, K. (2021). \emph{Dplyr: A grammar of data manipulation}. Retrieved from \url{https://CRAN.R-project.org/package=dplyr}

\leavevmode\hypertarget{ref-R-readr}{}%
Wickham, H., \& Hester, J. (2021). \emph{Readr: Read rectangular text data}. Retrieved from \url{https://CRAN.R-project.org/package=readr}

\leavevmode\hypertarget{ref-R-knitr}{}%
Xie, Y. (2015). \emph{Dynamic documents with {R} and knitr} (2nd ed.). Boca Raton, Florida: Chapman; Hall/CRC. Retrieved from \url{https://yihui.org/knitr/}

\leavevmode\hypertarget{ref-R-tinytex}{}%
Xie, Y. (2019). TinyTeX: A lightweight, cross-platform, and easy-to-maintain LaTeX distribution based on TeX live. \emph{TUGboat}, (1), 30--32. Retrieved from \url{http://tug.org/TUGboat/Contents/contents40-1.html}

\leavevmode\hypertarget{ref-R-rmarkdown_a}{}%
Xie, Y., Allaire, J. J., \& Grolemund, G. (2018). \emph{R markdown: The definitive guide}. Boca Raton, Florida: Chapman; Hall/CRC. Retrieved from \url{https://bookdown.org/yihui/rmarkdown}

\leavevmode\hypertarget{ref-R-DT}{}%
Xie, Y., Cheng, J., \& Tan, X. (2021). \emph{DT: A wrapper of the JavaScript library 'DataTables'}. Retrieved from \url{https://CRAN.R-project.org/package=DT}

\leavevmode\hypertarget{ref-R-rmarkdown_b}{}%
Xie, Y., Dervieux, C., \& Riederer, E. (2020). \emph{R markdown cookbook}. Boca Raton, Florida: Chapman; Hall/CRC. Retrieved from \url{https://bookdown.org/yihui/rmarkdown-cookbook}

\end{CSLReferences}

\endgroup


\end{document}
